\documentclass[11pt]{article}
\usepackage[utf8]{inputenc}
%Gummi|065|=)

\begin{document}
\subsection{Requerimientos}
A partir del enunciado del problema se extrajo los distintos requerimientos que debe cumplir el sistema, éstos son:
\begin{enumerate}
\item Mantener información de las sucursales junto con sus distintos servicios y a que empresa está asociado.
\item Mantener información de puntos de interés,  proporcionada por los turistas.
\item Obtener información adicional o modificar la existente desde los empresarios sobre sus propias sucursales.
\item Enriquecer la información de los puntos de interés acorde a la experiencia de los turistas.
\item Categorizar servicios de las distintas sucursales en el momento de agregar dicho servicio al sistema. Las categorías vienen predefinidas y el empresario puede asignar más de una categoría a un servicio.
\item Indicar dentro de la aplicación si un Empresario cuenta con el sello de turistmo sustentable.
\item Buscar puntos de interés registrados por medio de distintos criterios:
	\begin{itemize}
		\item Nombre sucursal.
		\item Servicios.
		\item Lugar geográfico.
	\end{itemize}
\item Enriquecer información existente sobre una sucursal por medio experiencia de turistas, estas son:
	\begin{itemize}
		\item 
		\item
		\item
	\end{itemize}

\item Permitir a turistas evaluar servicios de sucursales mediante a sistema de puntos de 1 a 5 y un comentario opcional.

\item Rutas a servicios.
\item Facilitar indicaciones a lugares de difícil acceso.
\item Indicaciones con fotografía de hitos y referencias conocidas.
\item Información de buses.
\item Proveer mapa turístico de oferta turística.
\item Crear itinerario.
\item Consultar itinerario.
\item Comentar itinerario.
\item Recomendaciones de itinerario de distinta duración.
\end{enumerate}




\end{document}
