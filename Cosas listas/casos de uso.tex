\documentclass[11pt]{article}
%Gummi|065|=)
\usepackage[utf8]{inputenc}
\begin{document}
%%%%%%%%%%%%%%%%%%%%%%%%%%%%%%%%%%%%%%%%%%%%%%%%%%%%%%%%%%%%%%%%%%%%%
%							CASOS DE USO							%
%%%%%%%%%%%%%%%%%%%%%%%%%%%%%%%%%%%%%%%%%%%%%%%%%%%%%%%%%%%%%%%%%%%%%
\section*{Casos de uso}
Para esta iteración, para cumplir los requerimientos acordados en un principio, se modificaron e implementaron los siguientes casos de uso:

\subsection*{\textbf{CASO DE USO UC1:} Ver mapa temático }
\textbf{Actor principal:} Usuario\\
\\
\textbf{Personal involucrado e intereses: }\\
\underline{Usuario:} Quiere visualizar un mapa de sucursales que tengan ciertos servicios.\\\underline{Google Maps:} Quiere recibir peticiones de PoI para mostrar en mapa de forma correcta.\\
\\
\textbf{Precondiciones:} El usuario apretó el botón de mapa temático\\
\\
\textbf{Postcondiciones:} El usuario visualiza un mapa (geográfico) con marcas donde se encuentran las sucursales que contienen los servicios que está buscando.\\
\\
\textbf{Escenario principal de éxito:}
\begin{enumerate}
\item Se envía una petición a Google Maps para proveer el mapa de la provincia de Arauco.
\item El usuario visualiza el mapa
\item El sistema invoca al caso de uso \textbf{Busar Servicios (UC2)}, lo que permite obtener una lista con sucursales filtrada.
\item Se envía una petición a Google Maps para que ubique las sucursales listadas en el mapa.
\item Las sucursales aparecen en el mapa.
\end{enumerate}
\textbf{Excepciones:}
\begin{enumerate}
\item[1-4'] Falla conexión con Google Maps, se muestra un mensaje de error.
\item[1-5'] El turista apreta cancelar.
\end{enumerate}
\subsection*{\textbf{CASO DE USO UC2:} Buscar sucursales }
\textbf{Actor principal:} Usuario\\
\\
\textbf{Personal involucrado e intereses: }\\\underline{Usuario:} Quiere encontrar sucursales que contenga servicios que busca.\\
\\
\textbf{Precondiciones:} El usuario necesita una lista filtrada de sucursales.\\
\\
\textbf{Postcondiciones:} EL usuario obtiene una lista de sucursales.\\
\\
\textbf{Escenario principal de éxito:}
\begin{enumerate}
\item El usuario apreta el botón de busqueda de servicios.
\item El usuario ingresa un valor en el campo de búsqueda.
\item El sistema busca sucursals con tipos de servicios acorde al campo de búsqueda ingresa.
\item Se muestra al usuario una lista con sucursales.
\end{enumerate}
\textbf{Excepciones:}
\begin{enumerate}
\item[2'a] El usuario cancela la operación.
\item[2'b] El usuario puede apretar el botón de busqueda con criterios específicos. El sistema invoca al caso de uso \textbf{Buscar por criterios específicos (UC3)}.
\item[4'] Lista vacía, se informa que no se encontró lo que el usuario buscaba.
\end{enumerate}
\subsection*{\textbf{CASO DE USO UC3:} Buscar por criterios específicos }
\textbf{Actor principal:} Usuario\\
\\
\textbf{Personal involucrado e intereses: }\\\underline{Usuario:} Quiere extender su busqueda de iteraciones o sucursales con mayor criterio de busqueda.
\\\\
\textbf{Precondiciones:}El usuario estaba buscando sucursales o itinerarios en el sistema, se informa que tipo de busqueda es.\\ 
\\
\textbf{Postcondiciones:} El usuario obtiene una lista con sucursales o itinerarios que cumplen con los criterios específicos.\\
\\
\textbf{Escenario principal de éxito:}
\begin{enumerate}
\item El usuario selecciona de una lista el/los criterios de busqueda que desea.
\item El usuario ingresa un valor de busqueda por cada criterio seleccionado.
\item El sistema busca los sucursales o itinerarios acorde al campo de búsqueda.
\item Se muestra al usuario una lista con sucursales o itinerarios.
\end{enumerate}
\textbf{Excepciones:}
\begin{enumerate}
\item[2'] El usuario puede cancelar la interacción.
\item[4'] Lista vacía, se informa que no se encontró lo que el usuario buscaba.
\end{enumerate}
\subsection*{\textbf{CASO DE USO UC4:} Buscar itinerarios }
\textbf{Actor principal:} Usuario\\
\\
\textbf{Personal involucrado e intereses: }\\\underline{Usuario} Quiere encntrar el/los itinerario/s que cumplan con su criterio de busqueda.\\
\\
\textbf{Precondiciones:} El usuario necesita una lista de itinerarios.\\
\\
\textbf{Postcondiciones:} El usuario obtiene una lista de itinerarios.\\
\\
\textbf{Escenario principal de éxito:}
\begin{enumerate}
\item Usuario apreta botón de búsqueda de itinerario.
\item Usuario ingresa valor en campo de búsqueda.
\item El sistema busca itinerarios acorde al campo de búsqueda.
\item Se muestra al usuario una lista con itinerarios.
\end{enumerate}
\textbf{Excepciones:}
\begin{enumerate}
\item[2'a] El usuario cancela la operación.
\item[2'b] El usuario puede apretar el botón de busqueda con criterios específicos. El sistema invoca al caso de uso \textbf{Buscar por criterios específicos (UC3)}.
\end{enumerate}
\subsection*{\textbf{CASO DE USO UC5:} Visualizar información de sucursal }
\textbf{Actor principal:} Usuario y Turista\\
\\
\textbf{Personal involucrado e intereses: }\\\underline{Usuario y Turista:} Quiere ver información asociada a una sucursal.\\
\\
\textbf{Precondiciones:} El usuario ha seleccionado una sucursal, ya sea desde busqueda, itinerario o mapa temático.\\
\\
\textbf{Postcondiciones:} El usuario ve en pantalla la información relevante de la sucursal.\\
\\
\textbf{Escenario principal de éxito:}
\begin{enumerate}
\item El sistema obtiene la información de la sucursal.
\item El sistema muestra la información al usuario.
\end{enumerate}
\textbf{Excepciones:}
\begin{enumerate}
\item[3'a] El turista apreta el botón de evaluar servicios, el sistema invoca al caso de uso \textbf{Evaluar Servicios (UC6)}. 
\item[3'b] El turista apreta el botón de añadir a itinerario, el sistemainvoac al caso de uso \textbf{Añadir a itinerario (UC7)}.
\item[3'c] El usuario apreta el botón de generar ruta, el sistema invoca al caso de uso \textbf{Generar ruta (UC8)}.
\item[3'd] El usuario apreta el botón de evaluar servicios o el botón de añadir a itinerario. Se muestra un mensaje de error por no iniciar sesión.
\end{enumerate}
\subsection*{\textbf{CASO DE USO UC6:} Evaluar servicio }
\textbf{Actor principal:} Turista\\
\\
\textbf{Personal involucrado e intereses: }\\\underline{Turista:} Desea evaluar servicios de una sucursal.
\\
\textbf{Precondiciones:} El turista se encuentra visualizando una sucursal.\\
\\
\textbf{Postcondiciones:} El turista ha evaluado los servicios que quería de una sucursal.\\
\\
\textbf{Escenario principal de éxito:}
\begin{enumerate}
\item El sistema muestra una lista de seriicios de la sucursa con estrellas sin nota o con nota si ya se había evaluado.
\item El turista selecciona cuantas estrellas le asigna a cada servicio que quiera.
\item El turista finaliza al presionar botón de evaluar.
\end{enumerate}
\textbf{Excepciones:}
\begin{enumerate}
\item[2'] El turista modifica una evaluación hecha previamente.
\item[1-3'] Turista Cancela.
\end{enumerate}

\subsection*{\textbf{CASO DE USO UC7:} Añadir a Itinerario }
\textbf{Actor principal:} Turista\\
\\
\textbf{Personal involucrado e intereses: }\\
\underline{Turista}: Desea añadir servicios a un itinerario creado por él.
\\\\
\textbf{Precondiciones:} El turista se encuentra visualizando una sucursal desde una lista o su detalle.\\
\\
\textbf{Postcondiciones:} El turista añade servicio(s) de una sucursal a un itinerario.\\
\\
\textbf{Esecnario principal de éxito:}
\begin{enumerate}
\item El turista selecciona en orden los servicios que desea agregar.
\item El turista apreta el botón agregar a itinerario.
\item EL sistema muestra una lista con itinerarios del turista.
\item El turista selecciona el itinerario al cual desea agregar servicios.
\item Finaliza apretando OK.
\end{enumerate}
\textbf{Excepciones:}
\begin{enumerate}
\item[1-5'] Cancelar.
\end{enumerate}



\subsection*{\textbf{CASO DE USO UC8:} Generar Ruta }
\textbf{Actor principal:} Usuario\\
\\
\textbf{Personal involucrado e intereses:}\\
\underline{Usuario}: Desea ver una ruta a una sucursal.\\
\underline{Google Maps}: Quiere recibir peticiones de ruta entre dos puntos en el mapa.
\\\\
\textbf{Precondiciones:} Usuario se encuentra visualizando sucursal desde una lista o el detalle y tiene activado el GPS.\\
\\
\textbf{Postcondiciones:} Usuario ve ruta desde su posición hasta una sucursal.\\
\\
\textbf{Esecnario principal de éxito:}
\begin{enumerate}
\item El usuario ingresa qué tipo de transporte utilizará.
\item Se envía información a MAPS junto con posición del usuaro, posición de sucursal y el medio de transporte.
\item MAPS entrega un mapa de la ruta.
\item El sistema muestra la información al usuario.
\end{enumerate}
\textbf{Excepciones:}
\begin{enumerate}
\item[2'] Falla la conexión con MAPS, se muestra una alerta.
\item[3'] No existe ruta, se entrega un mensaje de error.
\end{enumerate}


\subsection*{\textbf{CASO DE USO UC9:} Visualizar Itinerario }
\textbf{Actor principal:} Usuario, Turista\\
\\
\textbf{Personal involucrado e intereses: }\\
\underline{Usuario}: quiere visualizar los contenidos de un itinerario.\\
\underline{Turista}: puede visualizar el contenido de un itinerario y compartirlo. Además puede modificar o borrar itinerarios creados por él.
\\\\
\textbf{Precondiciones:} El usuario está visualizando una lista de itinerarios, por ejemplo después del caso de uso "buscar itinerario".\\
\\
\textbf{Postcondiciones:} El sistema muestra al usuario los detalles contenidos en un itinerario.\\
\\
\textbf{Esecnario principal de éxito:}
\begin{enumerate}
\item El usuario selecciona un itinerario de la lista de itinerarios.
\item El usuario selecciona la opción de visualización de contenido.
\item El sistema muestra la información detallada del itinerario.
\end{enumerate}
\textbf{Excepciones:}
\begin{enumerate}
\item[2'] El usuario deselecciona el itinerario y no se visualiza el contenido.
\item[3'a] El turista puede seleccionar la opción borrar y se salta al caso de uso "Borrar itinerario (CU10)".
\item[3'b] El turista puede seleccionar la opción modificar y se salta al caso de uso "Modificar itinerario (CU11)".
\item[3'c] El turista puede seleccionar la opción compartir y se salta al caso de uso "Compartir itinerario (CU12)".
\end{enumerate}


\subsection*{\textbf{CASO DE USO UC10: Borrar Itinerario}  }
\textbf{Actor principal:} Turista.\\
\\
\textbf{Personal involucrado e intereses: }\\
\underline{Turista}:Quiere eliminar un itinerario creado por él.
\\\\
\textbf{Precondiciones:} El turista ha seleccionado la opcion borrar itinerario desde la visualización de su itinerario.\\
\\
\textbf{Postcondiciones:} Se elimina el itinerario seleccionado de la base de datos.\\
\\
\textbf{Esecnario principal de éxito:}
\begin{enumerate}
\item El turista selecciona la opción de borrar itinerario.
\item Se elimina el itinerario seleccionado de la base de datos.
\end{enumerate}
\textbf{Excepciones:}
\begin{enumerate}
\item[1'] El turista puede cancelar la eliminación de su itinerario.
\end{enumerate}



\subsection*{\textbf{CASO DE USO UC11: Modificar Itinerario}  }
\textbf{Actor principal:} Turista.\\
\\
\textbf{Personal involucrado e intereses: }\\
\underline{Turista}:Quiere eliminar servicios contenidos en su itinerario o cambiar el orden de éstos.
\\\\
\textbf{Precondiciones:} El turista ha seleccionado la opcion modificar itinerario desde la visualización de su itinerario.\\
\\
\textbf{Postcondiciones:} Se modifica el contenido del itinerario.\\
\\
\textbf{Esecnario principal de éxito:}
\begin{enumerate}
\item El turista selecciona la opción de modificar itinerario.
\item El turista selecciona un servicio dentro del itinerario.
\item El turista selecciona otro servicio dentro del itinerario y se intercambian en el orden.
\end{enumerate}
\textbf{Excepciones:}
\begin{enumerate}
\item[3'a] El turista selecciona la opción borrar servicio para borrar el servicio seleccionado del itinerario.
\item[1-3'] El turista puede salir en cualquier momento del modo edición.
\end{enumerate}



\subsection*{\textbf{CASO DE USO UC12: Compartir Itinerario}  }
\textbf{Actor principal:} Turista.\\
\\
\textbf{Personal involucrado e intereses: }\\
\underline{Turista}:Quiere compartir un itinerario en redes sociales.
\\\\
\textbf{Precondiciones:} El turista está visualizando una lista de itinerarios, por ejemplo después del caso de uso "buscar itinerario" o está visualizando el contenido de un itinerario.\\
\\
\textbf{Postcondiciones:} El turista publica en la red social elegida el itinerario seleccionado.\\
\\
\textbf{Esecnario principal de éxito:}
\begin{enumerate}
\item El turista selecciona un itinerario de la lista de itinerarios.
\item El turista selecciona la opción compartir itinerario y selecciona dónde compartirlo.
\item Se abre la opción de compartir de la red social elegida.
\item Se comparte satisfactoriamente el itinerario seleccionado.
\end{enumerate}
\textbf{Excepciones:}
\begin{enumerate}
\item[2-3'] El turista puede cancelar la acción en todo momento.
\item[3']s Error al conectar con la red social, se cancela la acción.
\subsection*{\textbf{CASO DE USO UC13:} Crear itinerario }
\end{enumerate}
\textbf{Actor principal:} Turista\\
\\
\textbf{Personal involucrado e intereses: }\\\underline{Turista:} Quiere crear un itinerario en la aplicación.\\
\\
\textbf{Precondiciones:} El turista, desde sus itinerarios, accede a la creación de itinerarios.\\
\\
\textbf{Postcondiciones:} El turista crea un itinerario vacío.\\
\\
\textbf{Escenario principal de éxito:}
\\
\begin{enumerate}
\item El turista escribe el nombre del nuevo itinerario.
\item Si el nombre no se repite con otro, se crea el nuevo itinerario.
\end{enumerate}
\textbf{Excepciones:}
\begin{enumerate}
\item[1-2'] Cancelar: En cualquier momento el turista puede cancelar la operación.
\item[2'] El nombre del itinerario ya existe. Se emite un mensaje y se vuelve al paso 1.
\end{enumerate}
\end{document}
