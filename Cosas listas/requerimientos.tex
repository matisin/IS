\documentclass[11pt]{article}
\usepackage[utf8]{inputenc}
%Gummi|065|=)

\begin{document}
\subsection{Requerimientos}
A partir del enunciado del problema se extrajo los distintos requerimientos que debe cumplir el sistema, éstos son:
\begin{enumerate}
\item Mantener información de las sucursales junto con sus distintos servicios y a que empresa está asociado.
\item Obtener información adicional o modificar la existente desde los empresarios sobre sus propias sucursales y servicios.
\item Categorizar servicios de las distintas sucursales en el momento de agregar dicho servicio al sistema. Las categorías y sub-categorias vienen predefinidas en el sistema.
\item Indicar dentro de la aplicación si un Empresario cuenta con el sello de turistmo sustentable.
\item Buscar servicios registrados por medio de distintos criterios:
	\begin{itemize}
		\item Nombre Servicio.
		\item Punto de interés.
		\item Sucursal.
		\item Categoria.
		\item Sub-Categoria.
	\end{itemize}
\item Enriquecer información existente sobre una sucursal por medio experiencia de turistas, estas son:
	\begin{itemize}
		\item Comentarios
		\item Evaluación de distintos criterios.
		\item Indicaciones y consejos de orientación a una sucursal.
	\end{itemize}
\item Crear un sistema de 
\item Rutas a servicios.
\item Facilitar indicaciones a punto de interés de difícil acceso. mediante:
	\begin{itemize}
		\item Indicaciones con fotografía de hitos y referencias conocidas.
		\item Información de buses urbanos o lugares.
		\item Consejos de orientación proporcionada por
	\end{itemize}
\item Proveer mapa turístico de oferta turística categorizada.
\item Permitir a los turistas:
	\begin{itemize}
		\item Crear itinerario.
		\item Buscar itinerarios.
		\item Comentar itinerario.
		\item Borrar itinerario.
		\item Compartir itinerario en redes sociales externas.		
	\end{itemize}
\end{enumerate}
\end{document}
